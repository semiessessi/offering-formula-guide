The transliteration notation used in this book is a variant of the \textit{\indexed{Manuel de Codage}} (MdC), referring to an extended version of \textit{\indexed{Gardiner's sign list}}.

MdC is an Egyptological standard for computer encoding of hieroglyphic text based on the phonetic values. It was first published in 1988 and continues to be revised.

It uses a subset of the Latin alphabet, with capitalisation and Gardiner sign codes used for greater accuracy. It also includes markup characters to encode positioning and arrangement of hieroglyphs, although these will not be used in this book.

In this book a variation is used which prefers phonetic values to sign codes and adds some additional encodings. For convenience this variant is referred to as \indexed{MdC+}.

The Gardiner sign list categorises hieroglyphs into types, and assigns each one a unique code. Although a given \indexed{hieroglyph} my fit into multiple categories, they occur only once, and usually in the first applicable category. It was first published in 1927, although the discovery of new glyphs and variants has required its extension since then.

For example, the hieroglyph G10 portrays a \indexed{falcon}, representing the god \nname{Sokar}, standing on a \indexed{barque} on top of a sled.

\begin{figure} [H]
\centering
\includegraphics[width=0.5\textwidth]{../recoloured-tuxscribe-hieroglyphs/png/G10}
\caption{Hieroglyph G10}
\end{figure}

It could reasonably fit into the categories G (Birds), P (Ships and parts of ships), R (Temple furniture or sacred emblems) or U (Agriculture, crafts, and professions). It appears only once in the G category.

Transliterations used in this book will place in bold the parts to be vocalised. Any implied or otherwise absent hieroglyphs will be in curved brackets, and determinatives in square brackets. A dot is used to separate the pronoun suffices used in the language from the owning verb or noun.

Note that the following tables do not contain a comprehensive list of modern additions to the original \indexed{Gardiner's sign list}, nor is every hieroglyph listed here referred to later in this book.

\pagebreak

\section*{Manuel de Codage variant (MdC+)}
\markboth{Notation}{Manuel de Codage variant (MdC+)}
\addcontentsline{toc}{section}{Manuel de Codage variant (MdC+)}

These tables define the encoding used throughout this book for transliteration, and to render hieroglyphs in a way that can be vocalised.

\subsection*{Uniliteral signs}

Uniliteral signs represent individual sounds. They give us something like an alphabet, but are not used alone. In many cases they are used as phonetic compliments to assist in reading multiliteral signs.

They are presented separately to give the reader a clear impression of the set of sounds used in the ancient Egyptian language.

\begin{center}
\begin{tabularx}{\linewidth}{YYYY}
Hieroglyph & MdC+ & Transliteration & Gardiner code\\
\hline\\
\includegraphics[width=0.50\linewidth,height=0.50\linewidth,keepaspectratio]{../recoloured-tuxscribe-hieroglyphs/png/G1} & A & ꜣ & G1 \\ 
\vspace{0.30000cm} \includegraphics[width=0.50\linewidth,height=0.50\linewidth,keepaspectratio]{../recoloured-tuxscribe-hieroglyphs/png/D36} \vspace{0.30000cm} & a & ꜥ & D36 \\ 
\includegraphics[width=0.50\linewidth,height=0.50\linewidth,keepaspectratio]{../recoloured-tuxscribe-hieroglyphs/png/D58} & b & b & D58 \\ 
\vspace{0.12500cm} \includegraphics[width=0.50\linewidth,height=0.50\linewidth,keepaspectratio]{../recoloured-tuxscribe-hieroglyphs/png/D46} \vspace{0.12500cm} & d & d & D46 \\ 
\includegraphics[width=0.50\linewidth,height=0.50\linewidth,keepaspectratio]{../recoloured-tuxscribe-hieroglyphs/png/I10} & D & ḏ & I10 \\ 
\vspace{0.17500cm} \includegraphics[width=0.50\linewidth,height=0.50\linewidth,keepaspectratio]{../recoloured-tuxscribe-hieroglyphs/png/I9} \vspace{0.17500cm} & f & f & I9 \\ 
\includegraphics[width=0.35\linewidth,height=0.35\linewidth,keepaspectratio]{../recoloured-tuxscribe-hieroglyphs/png/W11} & g & g & W11 \\ 
\includegraphics[width=0.35\linewidth,height=0.35\linewidth,keepaspectratio]{../recoloured-tuxscribe-hieroglyphs/png/V33} & G & g & V33 \\ 
\includegraphics[width=0.35\linewidth,height=0.35\linewidth,keepaspectratio]{../recoloured-tuxscribe-hieroglyphs/png/O4} & h & h & O4 \\ 
\includegraphics[width=0.50\linewidth,height=0.50\linewidth,keepaspectratio]{../recoloured-tuxscribe-hieroglyphs/png/V28} & H & ḥ & V28 \\ 
\includegraphics[width=0.50\linewidth,height=0.50\linewidth,keepaspectratio]{../recoloured-tuxscribe-hieroglyphs/png/M17} & i & ꞽ & M17 \\ 
\vspace{0.30000cm} \includegraphics[width=0.50\linewidth,height=0.50\linewidth,keepaspectratio]{../recoloured-tuxscribe-hieroglyphs/png/V31} \vspace{0.30000cm} & k & k & V31 \\ 
\end{tabularx}
\end{center}


\begin{center}
\begin{tabularx}{\linewidth}{YYYY}
Hieroglyph & MdC+ & Transliteration & Gardiner code\\
\hline\\
\vspace{0.30000cm} \includegraphics[width=0.50\linewidth,height=0.50\linewidth,keepaspectratio]{../recoloured-tuxscribe-hieroglyphs/png/E23} \vspace{0.30000cm} & l & l & E23 \\ 
\includegraphics[width=0.50\linewidth,height=0.50\linewidth,keepaspectratio]{../recoloured-tuxscribe-hieroglyphs/png/G17} & m & m & G17 \\ 
\vspace{0.32500cm} \includegraphics[width=0.50\linewidth,height=0.50\linewidth,keepaspectratio]{../recoloured-tuxscribe-hieroglyphs/png/J15} \vspace{0.32500cm} & M & m & Aa15 \\ 
\vspace{0.50000cm} \includegraphics[width=0.50\linewidth,height=0.50\linewidth,keepaspectratio]{../recoloured-tuxscribe-hieroglyphs/png/N35} \vspace{0.50000cm} & n & n & N35 \\ 
\includegraphics[width=0.50\linewidth,height=0.50\linewidth,keepaspectratio]{../recoloured-tuxscribe-hieroglyphs/png/S3} & N & n & S3 \\ 
\includegraphics[width=0.25\linewidth,height=0.25\linewidth,keepaspectratio]{../recoloured-tuxscribe-hieroglyphs/png/Q3} & p & p & Q3 \\ 
\includegraphics[width=0.30\linewidth,height=0.30\linewidth,keepaspectratio]{../recoloured-tuxscribe-hieroglyphs/png/N29} & q & q & N29 \\ 
\vspace{0.30000cm} \includegraphics[width=0.50\linewidth,height=0.50\linewidth,keepaspectratio]{../recoloured-tuxscribe-hieroglyphs/png/D21} \vspace{0.30000cm} & r & r & D21 \\ 
\includegraphics[width=0.50\linewidth,height=0.50\linewidth,keepaspectratio]{../recoloured-tuxscribe-hieroglyphs/png/S29} & s & s & S29 \\ 
\includegraphics[width=0.50\linewidth,height=0.50\linewidth,keepaspectratio]{../recoloured-tuxscribe-hieroglyphs/png/N37} & S & š & N37 \\ 
\vspace{0.25000cm} \includegraphics[width=0.25\linewidth,height=0.25\linewidth,keepaspectratio]{../recoloured-tuxscribe-hieroglyphs/png/X1} \vspace{0.25000cm} & t & t & X1 \\ 
\vspace{0.37500cm} \includegraphics[width=0.50\linewidth,height=0.50\linewidth,keepaspectratio]{../recoloured-tuxscribe-hieroglyphs/png/V13} \vspace{0.37500cm} & T & ṯ & V13 \\ 
\includegraphics[width=0.50\linewidth,height=0.50\linewidth,keepaspectratio]{../recoloured-tuxscribe-hieroglyphs/png/G43} & w & w & G43 \\ 
\includegraphics[width=0.40\linewidth,height=0.40\linewidth,keepaspectratio]{../recoloured-tuxscribe-hieroglyphs/png/V1} & W & w & V1 \\ 
\includegraphics[width=0.25\linewidth,height=0.25\linewidth,keepaspectratio]{../recoloured-tuxscribe-hieroglyphs/png/J1} & x & ḫ & Aa1 \\ 
\vspace{0.32500cm} \includegraphics[width=0.50\linewidth,height=0.50\linewidth,keepaspectratio]{../recoloured-tuxscribe-hieroglyphs/png/F32} \vspace{0.32500cm} & X & ẖ & F32 \\ 
\includegraphics[width=0.50\linewidth,height=0.50\linewidth,keepaspectratio]{../recoloured-tuxscribe-hieroglyphs/png/M17A} & y & y & M17a \\ 
\includegraphics[width=0.25\linewidth,height=0.25\linewidth,keepaspectratio]{../recoloured-tuxscribe-hieroglyphs/png/Z4} & Y & y & Z4 \\ 
\vspace{0.32500cm} \includegraphics[width=0.50\linewidth,height=0.50\linewidth,keepaspectratio]{../recoloured-tuxscribe-hieroglyphs/png/O34} \vspace{0.32500cm} & z & z & O34 \\ 
\end{tabularx}
\end{center}


\subsection*{Multiliteral and other signs}

These signs are used for compounds of multiple sounds or as determinatives, numbers or grammatical hints. For example, the Z2 and Z3 hieroglyphs occurring early in the table are plural markers. The Z1 hieroglyph is used to denote the end of a word or to fill space between words for a more beautiful composition.

\begin{center}
	\begin{tabularx}{\linewidth}{YYYY}
		Hieroglyph & MdC+ & Transliteration & Gardiner code\\
		\hline\\
		\includegraphics[width=0.50\linewidth,height=0.50\linewidth,keepaspectratio]{../recoloured-tuxscribe-hieroglyphs/png/Z3} & {-}{-}{-} & {-}{-}{-} & Z3 \\ 
		\includegraphics[width=0.20\linewidth,height=0.20\linewidth,keepaspectratio]{../recoloured-tuxscribe-hieroglyphs/png/Z1} & | & | & Z1 \\ 
		\includegraphics[width=0.50\linewidth,height=0.50\linewidth,keepaspectratio]{../recoloured-tuxscribe-hieroglyphs/png/Z2} & ||| & ||| & Z2 \\ 
		\includegraphics[width=0.50\linewidth,height=0.50\linewidth,keepaspectratio]{../recoloured-tuxscribe-hieroglyphs/png/V20} & 10 & 10 & V20 \\ 
		\includegraphics[width=0.40\linewidth,height=0.40\linewidth,keepaspectratio]{../recoloured-tuxscribe-hieroglyphs/png/V1} & 100 & 100 & V1 \\ 
		\includegraphics[width=0.50\linewidth,height=0.50\linewidth,keepaspectratio]{../recoloured-tuxscribe-hieroglyphs/png/M12} & 1000 & 1000 & M12 \\ 
		\includegraphics[width=0.50\linewidth,height=0.50\linewidth,keepaspectratio]{../recoloured-tuxscribe-hieroglyphs/png/O29} & aA & ꜥꜣ & O29 \\ 
		\includegraphics[width=0.50\linewidth,height=0.50\linewidth,keepaspectratio]{../recoloured-tuxscribe-hieroglyphs/png/G2} & AA & ꜣꜣ & G2 \\ 
		\includegraphics[width=0.50\linewidth,height=0.50\linewidth,keepaspectratio]{../recoloured-tuxscribe-hieroglyphs/png/D59} & ab & ꜥb & D59 \\ 
		\includegraphics[width=0.50\linewidth,height=0.50\linewidth,keepaspectratio]{../recoloured-tuxscribe-hieroglyphs/png/U23} & Ab & ꜣb & U23 \\ 
		\includegraphics[width=0.50\linewidth,height=0.50\linewidth,keepaspectratio]{../recoloured-tuxscribe-hieroglyphs/png/S42} & abA & ꜥbꜣ & S42 \\ 
		\includegraphics[width=0.50\linewidth,height=0.50\linewidth,keepaspectratio]{../recoloured-tuxscribe-hieroglyphs/png/N11} & Abd & ꜣbd & N11 \\ 
		\includegraphics[width=0.50\linewidth,height=0.50\linewidth,keepaspectratio]{../recoloured-tuxscribe-hieroglyphs/png/E24} & Aby & ꜣby & E24 \\ 
	\end{tabularx}
\end{center}


\begin{center}
	\begin{tabularx}{\linewidth}{YYYY}
		Hieroglyph & MdC+ & Transliteration & Gardiner code\\
		\hline\\
		\includegraphics[width=0.50\linewidth,height=0.50\linewidth,keepaspectratio]{../recoloured-tuxscribe-hieroglyphs/png/K3} & ad & ꜥd & K3 \\ 
		\includegraphics[width=0.50\linewidth,height=0.50\linewidth,keepaspectratio]{../recoloured-tuxscribe-hieroglyphs/png/V26} & aD & ꜥḏ & V26 \\ 
		\includegraphics[width=0.50\linewidth,height=0.50\linewidth,keepaspectratio]{../recoloured-tuxscribe-hieroglyphs/png/T24} & aH & ꜥḥ & T24 \\ 
		\includegraphics[width=0.50\linewidth,height=0.50\linewidth,keepaspectratio]{../recoloured-tuxscribe-hieroglyphs/png/P6} & aHa & ꜥḥꜥ & P6 \\ 
		\includegraphics[width=0.50\linewidth,height=0.50\linewidth,keepaspectratio]{../recoloured-tuxscribe-hieroglyphs/png/D34} & aHA & ꜥḥꜣ & D34 \\ 
		\includegraphics[width=0.50\linewidth,height=0.50\linewidth,keepaspectratio]{../recoloured-tuxscribe-hieroglyphs/png/T12} & Ai & ꜣꞽ & T12 \\ 
		\includegraphics[width=0.50\linewidth,height=0.50\linewidth,keepaspectratio]{../recoloured-tuxscribe-hieroglyphs/png/S44} & Ams & ꜣms & S44 \\ 
		\includegraphics[width=0.50\linewidth,height=0.50\linewidth,keepaspectratio]{../recoloured-tuxscribe-hieroglyphs/png/S34} & anx & ꜥnḫ & S34 \\ 
		\includegraphics[width=0.50\linewidth,height=0.50\linewidth,keepaspectratio]{../recoloured-tuxscribe-hieroglyphs/png/S34} & Anx & ꜣnḫ & S34 \\ 
		\includegraphics[width=0.50\linewidth,height=0.50\linewidth,keepaspectratio]{../recoloured-tuxscribe-hieroglyphs/png/G35} & aq & ꜥq & G35 \\ 
		\includegraphics[width=0.50\linewidth,height=0.50\linewidth,keepaspectratio]{../recoloured-tuxscribe-hieroglyphs/png/T12} & Ar & ꜣr & T12 \\ 
		\includegraphics[width=0.50\linewidth,height=0.50\linewidth,keepaspectratio]{../recoloured-tuxscribe-hieroglyphs/png/V12} & arq & ꜥrq & V12 \\ 
		\includegraphics[width=0.50\linewidth,height=0.50\linewidth,keepaspectratio]{../recoloured-tuxscribe-hieroglyphs/png/Q1} & As & ꜣs & Q1 \\ 
	\end{tabularx}
\end{center}


\begin{center}
	\begin{tabularx}{\linewidth}{YYYY}
		Hieroglyph & MdC+ & Transliteration & Gardiner code\\
		\hline\\
		\includegraphics[width=0.50\linewidth,height=0.50\linewidth,keepaspectratio]{../recoloured-tuxscribe-hieroglyphs/png/I1} & aSA & ꜥšꜣ & I1 \\ 
		\includegraphics[width=0.50\linewidth,height=0.50\linewidth,keepaspectratio]{../recoloured-tuxscribe-hieroglyphs/png/S8} & Atf & ꜣtf & S8 \\ 
		\includegraphics[width=0.50\linewidth,height=0.50\linewidth,keepaspectratio]{../recoloured-tuxscribe-hieroglyphs/png/F40} & Aw & ꜣw & F40 \\ 
		\includegraphics[width=0.50\linewidth,height=0.50\linewidth,keepaspectratio]{../recoloured-tuxscribe-hieroglyphs/png/S39} & awt & ꜥwt & S39 \\ 
		\includegraphics[width=0.50\linewidth,height=0.50\linewidth,keepaspectratio]{../recoloured-tuxscribe-hieroglyphs/png/G25} & Ax & ꜣḫ & G25 \\ 
		\includegraphics[width=0.50\linewidth,height=0.50\linewidth,keepaspectratio]{../recoloured-tuxscribe-hieroglyphs/png/N27} & Axt & ꜣḫt & N27 \\ 
		\includegraphics[width=0.50\linewidth,height=0.50\linewidth,keepaspectratio]{../recoloured-tuxscribe-hieroglyphs/png/G29} & bA & bꜣ & G29 \\ 
		\includegraphics[width=0.50\linewidth,height=0.50\linewidth,keepaspectratio]{../recoloured-tuxscribe-hieroglyphs/png/G32} & baHi & bꜥḥꞽ & G32 \\ 
		\includegraphics[width=0.50\linewidth,height=0.50\linewidth,keepaspectratio]{../recoloured-tuxscribe-hieroglyphs/png/W2} & bAs & bꜣs & W2 \\ 
		\includegraphics[width=0.50\linewidth,height=0.50\linewidth,keepaspectratio]{../recoloured-tuxscribe-hieroglyphs/png/R9} & bd & bd & R9 \\ 
		\includegraphics[width=0.50\linewidth,height=0.50\linewidth,keepaspectratio]{../recoloured-tuxscribe-hieroglyphs/png/M34} & bdt & bdt & M34 \\ 
		\vspace{0.22500cm} \includegraphics[width=0.50\linewidth,height=0.50\linewidth,keepaspectratio]{../recoloured-tuxscribe-hieroglyphs/png/F18} \vspace{0.22500cm} & bH & bḥ & F18 \\ 
		\includegraphics[width=0.50\linewidth,height=0.50\linewidth,keepaspectratio]{../recoloured-tuxscribe-hieroglyphs/png/U16} & biA & bꞽꜣ & U16 \\ 
	\end{tabularx}
\end{center}


\begin{center}
	\begin{tabularx}{\linewidth}{YYYY}
		Hieroglyph & MdC+ & Transliteration & Gardiner code\\
		\hline\\
		\includegraphics[width=0.50\linewidth,height=0.50\linewidth,keepaspectratio]{../recoloured-tuxscribe-hieroglyphs/png/L2} & bit & bꞽt & L2 \\ 
		\includegraphics[width=0.50\linewidth,height=0.50\linewidth,keepaspectratio]{../recoloured-tuxscribe-hieroglyphs/png/M30} & bnr & bnr & M30 \\ 
		\includegraphics[width=0.50\linewidth,height=0.50\linewidth,keepaspectratio]{../recoloured-tuxscribe-hieroglyphs/png/K5} & bz & bz & K5 \\ 
		\includegraphics[width=0.50\linewidth,height=0.50\linewidth,keepaspectratio]{../recoloured-tuxscribe-hieroglyphs/png/U28} & DA & ḏꜣ & U28 \\ 
		\includegraphics[width=0.50\linewidth,height=0.50\linewidth,keepaspectratio]{../recoloured-tuxscribe-hieroglyphs/png/S41} & Dam & ḏꜥm & S41 \\ 
		\includegraphics[width=0.50\linewidth,height=0.50\linewidth,keepaspectratio]{../recoloured-tuxscribe-hieroglyphs/png/F16} & db & db & F16 \\ 
		\includegraphics[width=0.50\linewidth,height=0.50\linewidth,keepaspectratio]{../recoloured-tuxscribe-hieroglyphs/png/G22} & Db & ḏb & G22 \\ 
		\includegraphics[width=0.50\linewidth,height=0.50\linewidth,keepaspectratio]{../recoloured-tuxscribe-hieroglyphs/png/D50} & Dba & ḏbꜥ & D50 \\ 
		\includegraphics[width=0.50\linewidth,height=0.50\linewidth,keepaspectratio]{../recoloured-tuxscribe-hieroglyphs/png/T25} & DbA & ḏbꜣ & T25 \\ 
		\includegraphics[width=0.50\linewidth,height=0.50\linewidth,keepaspectratio]{../recoloured-tuxscribe-hieroglyphs/png/R11} & dd & dd & R11 \\ 
		\includegraphics[width=0.50\linewidth,height=0.50\linewidth,keepaspectratio]{../recoloured-tuxscribe-hieroglyphs/png/R11} & Dd & ḏd & R11 \\ 
		\includegraphics[width=0.50\linewidth,height=0.50\linewidth,keepaspectratio]{../recoloured-tuxscribe-hieroglyphs/png/I11} & DD & ḏḏ & I11 \\ 
		\includegraphics[width=0.50\linewidth,height=0.50\linewidth,keepaspectratio]{../recoloured-tuxscribe-hieroglyphs/png/C3} & DHwty & ḏḥwty & C3 \\ 
	\end{tabularx}
\end{center}


\begin{center}
	\begin{tabularx}{\linewidth}{YYYY}
		Hieroglyph & MdC+ & Transliteration & Gardiner code\\
		\hline\\
		\includegraphics[width=0.50\linewidth,height=0.50\linewidth,keepaspectratio]{../recoloured-tuxscribe-hieroglyphs/png/X8} & di & dꞽ & X8 \\ 
		\includegraphics[width=0.50\linewidth,height=0.50\linewidth,keepaspectratio]{../recoloured-tuxscribe-hieroglyphs/png/S23} & dmD & dmḏ & S23 \\ 
		\includegraphics[width=0.50\linewidth,height=0.50\linewidth,keepaspectratio]{../recoloured-tuxscribe-hieroglyphs/png/M36} & Dr & ḏr & M36 \\ 
		\includegraphics[width=0.30\linewidth,height=0.30\linewidth,keepaspectratio]{../recoloured-tuxscribe-hieroglyphs/png/F21} & DrD & ḏrḏ & F21 \\ 
		\includegraphics[width=0.50\linewidth,height=0.50\linewidth,keepaspectratio]{../recoloured-tuxscribe-hieroglyphs/png/G27} & dSr & dšr & G27 \\ 
		\includegraphics[width=0.50\linewidth,height=0.50\linewidth,keepaspectratio]{../recoloured-tuxscribe-hieroglyphs/png/D45} & Dsr & ḏsr & D45 \\ 
		\includegraphics[width=0.50\linewidth,height=0.50\linewidth,keepaspectratio]{../recoloured-tuxscribe-hieroglyphs/png/S3} & dSrt & dšrt & S3 \\ 
		\includegraphics[width=0.50\linewidth,height=0.50\linewidth,keepaspectratio]{../recoloured-tuxscribe-hieroglyphs/png/N26} & Dw & ḏw & N26 \\ 
		\includegraphics[width=0.50\linewidth,height=0.50\linewidth,keepaspectratio]{../recoloured-tuxscribe-hieroglyphs/png/N14} & dwA & dwꜣ & N14 \\ 
		\includegraphics[width=0.50\linewidth,height=0.50\linewidth,keepaspectratio]{../recoloured-tuxscribe-hieroglyphs/png/N15} & dwAt & dwꜣt & N15 \\ 
		\includegraphics[width=0.50\linewidth,height=0.50\linewidth,keepaspectratio]{../recoloured-tuxscribe-hieroglyphs/png/D19} & fnD & fnḏ & D19 \\ 
		\includegraphics[width=0.50\linewidth,height=0.50\linewidth,keepaspectratio]{../recoloured-tuxscribe-hieroglyphs/png/G38} & gb & gb & G38 \\ 
		\includegraphics[width=0.50\linewidth,height=0.50\linewidth,keepaspectratio]{../recoloured-tuxscribe-hieroglyphs/png/D56} & gH & gḥ & D56 \\ 
	\end{tabularx}
\end{center}


\begin{center}
	\begin{tabularx}{\linewidth}{YYYY}
		Hieroglyph & MdC+ & Transliteration & Gardiner code\\
		\hline\\
		\includegraphics[width=0.50\linewidth,height=0.50\linewidth,keepaspectratio]{../recoloured-tuxscribe-hieroglyphs/png/D56} & gHs & gḥs & D56 \\ 
		\includegraphics[width=0.50\linewidth,height=0.50\linewidth,keepaspectratio]{../recoloured-tuxscribe-hieroglyphs/png/G28} & gm & gm & G28 \\ 
		\includegraphics[width=0.50\linewidth,height=0.50\linewidth,keepaspectratio]{../recoloured-tuxscribe-hieroglyphs/png/U17} & grg & grg & U17 \\ 
		\includegraphics[width=0.50\linewidth,height=0.50\linewidth,keepaspectratio]{../recoloured-tuxscribe-hieroglyphs/png/F4} & HA & ḥꜣ & F4 \\ 
		\includegraphics[width=0.50\linewidth,height=0.50\linewidth,keepaspectratio]{../recoloured-tuxscribe-hieroglyphs/png/F4} & HAt & ḥꜣt & F4 \\ 
		\includegraphics[width=0.50\linewidth,height=0.50\linewidth,keepaspectratio]{../recoloured-tuxscribe-hieroglyphs/png/U13} & hb & hb & U13 \\ 
		\includegraphics[width=0.50\linewidth,height=0.50\linewidth,keepaspectratio]{../recoloured-tuxscribe-hieroglyphs/png/W3} & Hb & ḥb & W3 \\ 
		\includegraphics[width=0.50\linewidth,height=0.50\linewidth,keepaspectratio]{../recoloured-tuxscribe-hieroglyphs/png/T3} & HD & ḥḏ & T3 \\ 
		\includegraphics[width=0.50\linewidth,height=0.50\linewidth,keepaspectratio]{../recoloured-tuxscribe-hieroglyphs/png/T6} & HDD & ḥḏḏ & T6 \\ 
		\includegraphics[width=0.50\linewidth,height=0.50\linewidth,keepaspectratio]{../recoloured-tuxscribe-hieroglyphs/png/S1} & HDt & ḥḏt & S1 \\ 
		\includegraphics[width=0.50\linewidth,height=0.50\linewidth,keepaspectratio]{../recoloured-tuxscribe-hieroglyphs/png/I8} & Hfn & ḥfn & I8 \\ 
		\includegraphics[width=0.50\linewidth,height=0.50\linewidth,keepaspectratio]{../recoloured-tuxscribe-hieroglyphs/png/C11} & HH & ḥḥ & C11 \\ 
		\includegraphics[width=0.50\linewidth,height=0.50\linewidth,keepaspectratio]{../recoloured-tuxscribe-hieroglyphs/png/N42} & Hm & ḥm & N42 \\ 
	\end{tabularx}
\end{center}


\begin{center}
	\begin{tabularx}{\linewidth}{YYYY}
		Hieroglyph & MdC+ & Transliteration & Gardiner code\\
		\hline\\
		\includegraphics[width=0.50\linewidth,height=0.50\linewidth,keepaspectratio]{../recoloured-tuxscribe-hieroglyphs/png/U24} & Hmt & ḥmt & U24 \\ 
		\includegraphics[width=0.50\linewidth,height=0.50\linewidth,keepaspectratio]{../recoloured-tuxscribe-hieroglyphs/png/U24} & Hmwt & ḥmwt & U24 \\ 
		\includegraphics[width=0.50\linewidth,height=0.50\linewidth,keepaspectratio]{../recoloured-tuxscribe-hieroglyphs/png/M2} & Hn & ḥn & M2 \\ 
		\includegraphics[width=0.50\linewidth,height=0.50\linewidth,keepaspectratio]{../recoloured-tuxscribe-hieroglyphs/png/N8} & Hnmmt & ḥnmmt & N8 \\ 
		\includegraphics[width=0.30\linewidth,height=0.30\linewidth,keepaspectratio]{../recoloured-tuxscribe-hieroglyphs/png/W22} & Hnqt & ḥnqt & W22 \\ 
		\includegraphics[width=0.50\linewidth,height=0.50\linewidth,keepaspectratio]{../recoloured-tuxscribe-hieroglyphs/png/S38} & HqA & ḥqꜣ & S38 \\ 
		\includegraphics[width=0.50\linewidth,height=0.50\linewidth,keepaspectratio]{../recoloured-tuxscribe-hieroglyphs/png/U11} & HqAt & ḥqꜣt & U11 \\ 
		\includegraphics[width=0.30\linewidth,height=0.30\linewidth,keepaspectratio]{../recoloured-tuxscribe-hieroglyphs/png/D2} & Hr & ḥr & D2 \\ 
		\includegraphics[width=0.30\linewidth,height=0.30\linewidth,keepaspectratio]{../recoloured-tuxscribe-hieroglyphs/png/N5} & hrw & hrw & N5 \\ 
		\vspace{0.22500cm} \includegraphics[width=0.50\linewidth,height=0.50\linewidth,keepaspectratio]{../recoloured-tuxscribe-hieroglyphs/png/R4} \vspace{0.22500cm} & Htp & ḥtp & R4 \\ 
		\vspace{0.22500cm} \includegraphics[width=0.50\linewidth,height=0.50\linewidth,keepaspectratio]{../recoloured-tuxscribe-hieroglyphs/png/F18} \vspace{0.22500cm} & Hw & ḥw & F18 \\ 
		\includegraphics[width=0.50\linewidth,height=0.50\linewidth,keepaspectratio]{../recoloured-tuxscribe-hieroglyphs/png/O6} & Hwt & ḥwt & O6 \\ 
		\includegraphics[width=0.50\linewidth,height=0.50\linewidth,keepaspectratio]{../recoloured-tuxscribe-hieroglyphs/png/W14} & Hz & ḥz & W14 \\ 
	\end{tabularx}
\end{center}


\begin{center}
	\begin{tabularx}{\linewidth}{YYYY}
		Hieroglyph & MdC+ & Transliteration & Gardiner code\\
		\hline\\
		\includegraphics[width=0.50\linewidth,height=0.50\linewidth,keepaspectratio]{../recoloured-tuxscribe-hieroglyphs/png/W10} & iab & ꞽꜥb & W10 \\ 
		\includegraphics[width=0.50\linewidth,height=0.50\linewidth,keepaspectratio]{../recoloured-tuxscribe-hieroglyphs/png/R15} & iAb & ꞽꜣb & R15 \\ 
		\includegraphics[width=0.50\linewidth,height=0.50\linewidth,keepaspectratio]{../recoloured-tuxscribe-hieroglyphs/png/N4} & iAdt & ꞽꜣdt & N4 \\ 
		\includegraphics[width=0.50\linewidth,height=0.50\linewidth,keepaspectratio]{../recoloured-tuxscribe-hieroglyphs/png/N11} & iaH & ꞽꜥḥ & N11 \\ 
		\includegraphics[width=0.50\linewidth,height=0.50\linewidth,keepaspectratio]{../recoloured-tuxscribe-hieroglyphs/png/M1} & iAm & ꞽꜣm & M1 \\ 
		\includegraphics[width=0.50\linewidth,height=0.50\linewidth,keepaspectratio]{../recoloured-tuxscribe-hieroglyphs/png/N30} & iAt & ꞽꜣt & N30 \\ 
		\includegraphics[width=0.50\linewidth,height=0.50\linewidth,keepaspectratio]{../recoloured-tuxscribe-hieroglyphs/png/F34} & ib & ꞽb & F34 \\ 
		\includegraphics[width=0.50\linewidth,height=0.50\linewidth,keepaspectratio]{../recoloured-tuxscribe-hieroglyphs/png/Y6} & ibA & ꞽbꜣ & Y6 \\ 
		\includegraphics[width=0.30\linewidth,height=0.30\linewidth,keepaspectratio]{../recoloured-tuxscribe-hieroglyphs/png/N41} & id & ꞽd & N41 \\ 
		\includegraphics[width=0.30\linewidth,height=0.30\linewidth,keepaspectratio]{../recoloured-tuxscribe-hieroglyphs/png/F21} & idn & ꞽdn & F21 \\ 
		\includegraphics[width=0.35\linewidth,height=0.35\linewidth,keepaspectratio]{../recoloured-tuxscribe-hieroglyphs/png/V37} & idr & ꞽdr & V37 \\ 
		\includegraphics[width=0.50\linewidth,height=0.50\linewidth,keepaspectratio]{../recoloured-tuxscribe-hieroglyphs/png/N4} & idt & ꞽdt & N4 \\ 
		\includegraphics[width=0.50\linewidth,height=0.50\linewidth,keepaspectratio]{../recoloured-tuxscribe-hieroglyphs/png/T24} & iH & ꞽḥ & T24 \\ 
	\end{tabularx}
\end{center}


\begin{center}
	\begin{tabularx}{\linewidth}{YYYY}
		Hieroglyph & MdC+ & Transliteration & Gardiner code\\
		\hline\\
		\includegraphics[width=0.50\linewidth,height=0.50\linewidth,keepaspectratio]{../recoloured-tuxscribe-hieroglyphs/png/J13} & im & ꞽm & Aa13 \\ 
		\includegraphics[width=0.50\linewidth,height=0.50\linewidth,keepaspectratio]{../recoloured-tuxscribe-hieroglyphs/png/Z11} & Im & Im & Z11 \\ 
		\includegraphics[width=0.50\linewidth,height=0.50\linewidth,keepaspectratio]{../recoloured-tuxscribe-hieroglyphs/png/F39} & imAx & ꞽmꜣḫ & F39 \\ 
		\includegraphics[width=0.50\linewidth,height=0.50\linewidth,keepaspectratio]{../recoloured-tuxscribe-hieroglyphs/png/Z11} & imi & ꞽmꞽ & Z11 \\ 
		\includegraphics[width=0.50\linewidth,height=0.50\linewidth,keepaspectratio]{../recoloured-tuxscribe-hieroglyphs/png/R14} & imnt & ꞽmnt & R14 \\ 
		\includegraphics[width=0.50\linewidth,height=0.50\linewidth,keepaspectratio]{../recoloured-tuxscribe-hieroglyphs/png/K1} & in & ꞽn & K1 \\ 
		\includegraphics[width=0.50\linewidth,height=0.50\linewidth,keepaspectratio]{../recoloured-tuxscribe-hieroglyphs/png/O36} & inb & ꞽnb & O36 \\ 
		\includegraphics[width=0.50\linewidth,height=0.50\linewidth,keepaspectratio]{../recoloured-tuxscribe-hieroglyphs/png/W25} & ini & ꞽnꞽ & W25 \\ 
		\includegraphics[width=0.50\linewidth,height=0.50\linewidth,keepaspectratio]{../recoloured-tuxscribe-hieroglyphs/png/C6} & inpw & ꞽnpw & C6 \\ 
		\includegraphics[width=0.25\linewidth,height=0.25\linewidth,keepaspectratio]{../recoloured-tuxscribe-hieroglyphs/png/O45} & ipt & ꞽpt & O45 \\ 
		\vspace{0.06250cm} \includegraphics[width=0.50\linewidth,height=0.50\linewidth,keepaspectratio]{../recoloured-tuxscribe-hieroglyphs/png/D4} \vspace{0.06250cm} & ir & ꞽr & D4 \\ 
		\vspace{0.06250cm} \includegraphics[width=0.50\linewidth,height=0.50\linewidth,keepaspectratio]{../recoloured-tuxscribe-hieroglyphs/png/D4} \vspace{0.06250cm} & iri & ꞽrꞽ & D4 \\ 
		\includegraphics[width=0.50\linewidth,height=0.50\linewidth,keepaspectratio]{../recoloured-tuxscribe-hieroglyphs/png/A47} & iry & ꞽry & A47 \\ 
	\end{tabularx}
\end{center}


\begin{center}
	\begin{tabularx}{\linewidth}{YYYY}
		Hieroglyph & MdC+ & Transliteration & Gardiner code\\
		\hline\\
		\includegraphics[width=0.50\linewidth,height=0.50\linewidth,keepaspectratio]{../recoloured-tuxscribe-hieroglyphs/png/F44} & isw & ꞽsw & F44 \\ 
		\includegraphics[width=0.50\linewidth,height=0.50\linewidth,keepaspectratio]{../recoloured-tuxscribe-hieroglyphs/png/U10} & it & ꞽt & U10 \\ 
		\includegraphics[width=0.50\linewidth,height=0.50\linewidth,keepaspectratio]{../recoloured-tuxscribe-hieroglyphs/png/V15} & iTi & ꞽṯꞽ & V15 \\ 
		\includegraphics[width=0.50\linewidth,height=0.50\linewidth,keepaspectratio]{../recoloured-tuxscribe-hieroglyphs/png/N18} & iw & ꞽw & N18 \\ 
		\includegraphics[width=0.50\linewidth,height=0.50\linewidth,keepaspectratio]{../recoloured-tuxscribe-hieroglyphs/png/F44} & iwa & ꞽwꜥ & F44 \\ 
		\includegraphics[width=0.50\linewidth,height=0.50\linewidth,keepaspectratio]{../recoloured-tuxscribe-hieroglyphs/png/O28} & iwn & ꞽwn & O28 \\ 
		\includegraphics[width=0.50\linewidth,height=0.50\linewidth,keepaspectratio]{../recoloured-tuxscribe-hieroglyphs/png/M40} & iz & ꞽz & M40 \\ 
		\includegraphics[width=0.35\linewidth,height=0.35\linewidth,keepaspectratio]{../recoloured-tuxscribe-hieroglyphs/png/D28} & kA & kꜣ & D28 \\ 
		\includegraphics[width=0.50\linewidth,height=0.50\linewidth,keepaspectratio]{../recoloured-tuxscribe-hieroglyphs/png/R5} & kAp & kꜣp & R5 \\ 
		\includegraphics[width=0.50\linewidth,height=0.50\linewidth,keepaspectratio]{../recoloured-tuxscribe-hieroglyphs/png/O18} & kAr & kꜣr & O18 \\ 
		\includegraphics[width=0.50\linewidth,height=0.50\linewidth,keepaspectratio]{../recoloured-tuxscribe-hieroglyphs/png/F22} & kfA & kfꜣ & F22 \\ 
		\includegraphics[width=0.25\linewidth,height=0.25\linewidth,keepaspectratio]{../recoloured-tuxscribe-hieroglyphs/png/I6} & km & km & I6 \\ 
		\includegraphics[width=0.50\linewidth,height=0.50\linewidth,keepaspectratio]{../recoloured-tuxscribe-hieroglyphs/png/R5} & kp & kp & R5 \\ 
	\end{tabularx}
\end{center}


\begin{center}
	\begin{tabularx}{\linewidth}{YYYY}
		Hieroglyph & MdC+ & Transliteration & Gardiner code\\
		\hline\\
		\includegraphics[width=0.50\linewidth,height=0.50\linewidth,keepaspectratio]{../recoloured-tuxscribe-hieroglyphs/png/U2} & mA & mꜣ & U2 \\ 
		\includegraphics[width=0.50\linewidth,height=0.50\linewidth,keepaspectratio]{../recoloured-tuxscribe-hieroglyphs/png/U1} & MA & mꜣ & U1 \\ 
		\includegraphics[width=0.50\linewidth,height=0.50\linewidth,keepaspectratio]{../recoloured-tuxscribe-hieroglyphs/png/C10} & mAat & mꜣꜥt & C10 \\ 
		\includegraphics[width=0.50\linewidth,height=0.50\linewidth,keepaspectratio]{../recoloured-tuxscribe-hieroglyphs/png/E22} & mAi & mꜣꞽ & E22 \\ 
		\includegraphics[width=0.50\linewidth,height=0.50\linewidth,keepaspectratio]{../recoloured-tuxscribe-hieroglyphs/png/G46} & mAw & mꜣw & G46 \\ 
		\includegraphics[width=0.50\linewidth,height=0.50\linewidth,keepaspectratio]{../recoloured-tuxscribe-hieroglyphs/png/S43} & md & md & S43 \\ 
		\includegraphics[width=0.50\linewidth,height=0.50\linewidth,keepaspectratio]{../recoloured-tuxscribe-hieroglyphs/png/V20} & mD & mḏ & V20 \\ 
		\includegraphics[width=0.50\linewidth,height=0.50\linewidth,keepaspectratio]{../recoloured-tuxscribe-hieroglyphs/png/Y1} & mDAt & mḏꜣt & Y1 \\ 
		\includegraphics[width=0.50\linewidth,height=0.50\linewidth,keepaspectratio]{../recoloured-tuxscribe-hieroglyphs/png/S10} & mDH & mḏḥ & S10 \\ 
		\includegraphics[width=0.50\linewidth,height=0.50\linewidth,keepaspectratio]{../recoloured-tuxscribe-hieroglyphs/png/V19} & mDt & mḏt & V19 \\ 
		\includegraphics[width=0.50\linewidth,height=0.50\linewidth,keepaspectratio]{../recoloured-tuxscribe-hieroglyphs/png/V22} & mH & mḥ & V22 \\ 
		\includegraphics[width=0.50\linewidth,height=0.50\linewidth,keepaspectratio]{../recoloured-tuxscribe-hieroglyphs/png/N36} & mi & mꞽ & N36 \\ 
		\includegraphics[width=0.50\linewidth,height=0.50\linewidth,keepaspectratio]{../recoloured-tuxscribe-hieroglyphs/png/W19} & Mi & mꞽ & W19 \\ 
	\end{tabularx}
\end{center}


\begin{center}
	\begin{tabularx}{\linewidth}{YYYY}
		Hieroglyph & MdC+ & Transliteration & Gardiner code\\
		\hline\\
		\includegraphics[width=0.50\linewidth,height=0.50\linewidth,keepaspectratio]{../recoloured-tuxscribe-hieroglyphs/png/G18} & mm & mm & G18 \\ 
		\includegraphics[width=0.50\linewidth,height=0.50\linewidth,keepaspectratio]{../recoloured-tuxscribe-hieroglyphs/png/Y5} & mn & mn & Y5 \\ 
		\includegraphics[width=0.40\linewidth,height=0.40\linewidth,keepaspectratio]{../recoloured-tuxscribe-hieroglyphs/png/D27} & mnD & mnḏ & D27 \\ 
		\includegraphics[width=0.50\linewidth,height=0.50\linewidth,keepaspectratio]{../recoloured-tuxscribe-hieroglyphs/png/Y3} & mnhd & mnhd & Y3 \\ 
		\includegraphics[width=0.50\linewidth,height=0.50\linewidth,keepaspectratio]{../recoloured-tuxscribe-hieroglyphs/png/S18} & mnit & mnꞽt & S18 \\ 
		\includegraphics[width=0.50\linewidth,height=0.50\linewidth,keepaspectratio]{../recoloured-tuxscribe-hieroglyphs/png/A33} & mniw & mnꞽw & A33 \\ 
		\includegraphics[width=0.50\linewidth,height=0.50\linewidth,keepaspectratio]{../recoloured-tuxscribe-hieroglyphs/png/C8} & mnw & mnw & C8 \\ 
		\includegraphics[width=0.50\linewidth,height=0.50\linewidth,keepaspectratio]{../recoloured-tuxscribe-hieroglyphs/png/U22} & mnx & mnḫ & U22 \\ 
		\includegraphics[width=0.50\linewidth,height=0.50\linewidth,keepaspectratio]{../recoloured-tuxscribe-hieroglyphs/png/S27} & mnxt & mnḫt & S27 \\ 
		\includegraphics[width=0.50\linewidth,height=0.50\linewidth,keepaspectratio]{../recoloured-tuxscribe-hieroglyphs/png/U7} & mr & mr & U7 \\ 
		\includegraphics[width=0.50\linewidth,height=0.50\linewidth,keepaspectratio]{../recoloured-tuxscribe-hieroglyphs/png/U6} & Mr & mr & U6 \\ 
		\includegraphics[width=0.50\linewidth,height=0.50\linewidth,keepaspectratio]{../recoloured-tuxscribe-hieroglyphs/png/N36} & mri & mrꞽ & N36 \\ 
		\includegraphics[width=0.50\linewidth,height=0.50\linewidth,keepaspectratio]{../recoloured-tuxscribe-hieroglyphs/png/F31} & ms & ms & F31 \\ 
	\end{tabularx}
\end{center}


\begin{center}
	\begin{tabularx}{\linewidth}{YYYY}
		Hieroglyph & MdC+ & Transliteration & Gardiner code\\
		\hline\\
		\includegraphics[width=0.50\linewidth,height=0.50\linewidth,keepaspectratio]{../recoloured-tuxscribe-hieroglyphs/png/A12} & mSa & mšꜥ & A12 \\ 
		\includegraphics[width=0.30\linewidth,height=0.30\linewidth,keepaspectratio]{../recoloured-tuxscribe-hieroglyphs/png/F21} & msDr & msḏr & F21 \\ 
		\includegraphics[width=0.50\linewidth,height=0.50\linewidth,keepaspectratio]{../recoloured-tuxscribe-hieroglyphs/png/B3} & msi & msꞽ & B3 \\ 
		\includegraphics[width=0.50\linewidth,height=0.50\linewidth,keepaspectratio]{../recoloured-tuxscribe-hieroglyphs/png/V32} & msn & msn & V32 \\ 
		\includegraphics[width=0.50\linewidth,height=0.50\linewidth,keepaspectratio]{../recoloured-tuxscribe-hieroglyphs/png/D52} & mt & mt & D52 \\ 
		\includegraphics[width=0.50\linewidth,height=0.50\linewidth,keepaspectratio]{../recoloured-tuxscribe-hieroglyphs/png/N35A} & mw & mw & N35a \\ 
		\includegraphics[width=0.50\linewidth,height=0.50\linewidth,keepaspectratio]{../recoloured-tuxscribe-hieroglyphs/png/G14} & mwt & mwt & G14 \\ 
		\includegraphics[width=0.50\linewidth,height=0.50\linewidth,keepaspectratio]{../recoloured-tuxscribe-hieroglyphs/png/U38} & mxAt & mḫꜣt & U38 \\ 
		\includegraphics[width=0.50\linewidth,height=0.50\linewidth,keepaspectratio]{../recoloured-tuxscribe-hieroglyphs/png/I3} & mzH & mzḥ & I3 \\ 
		\includegraphics[width=0.50\linewidth,height=0.50\linewidth,keepaspectratio]{../recoloured-tuxscribe-hieroglyphs/png/V30} & nb & nb & V30 \\ 
		\includegraphics[width=0.50\linewidth,height=0.50\linewidth,keepaspectratio]{../recoloured-tuxscribe-hieroglyphs/png/G16} & nbty & nbty & G16 \\ 
		\includegraphics[width=0.50\linewidth,height=0.50\linewidth,keepaspectratio]{../recoloured-tuxscribe-hieroglyphs/png/S12} & nbw & nbw & S12 \\ 
		\includegraphics[width=0.50\linewidth,height=0.50\linewidth,keepaspectratio]{../recoloured-tuxscribe-hieroglyphs/png/J27} & nD & nḏ & Aa27 \\ 
	\end{tabularx}
\end{center}


\begin{center}
	\begin{tabularx}{\linewidth}{YYYY}
		Hieroglyph & MdC+ & Transliteration & Gardiner code\\
		\hline\\
		\includegraphics[width=0.50\linewidth,height=0.50\linewidth,keepaspectratio]{../recoloured-tuxscribe-hieroglyphs/png/M29} & nDm & nḏm & M29 \\ 
		\includegraphics[width=0.50\linewidth,height=0.50\linewidth,keepaspectratio]{../recoloured-tuxscribe-hieroglyphs/png/G37} & nDs & nḏs & G37 \\ 
		\includegraphics[width=0.50\linewidth,height=0.50\linewidth,keepaspectratio]{../recoloured-tuxscribe-hieroglyphs/png/F35} & nfr & nfr & F35 \\ 
		\includegraphics[width=0.50\linewidth,height=0.50\linewidth,keepaspectratio]{../recoloured-tuxscribe-hieroglyphs/png/P5} & nfw & nfw & P5 \\ 
		\includegraphics[width=0.50\linewidth,height=0.50\linewidth,keepaspectratio]{../recoloured-tuxscribe-hieroglyphs/png/G21} & nH & nḥ & G21 \\ 
		\includegraphics[width=0.50\linewidth,height=0.50\linewidth,keepaspectratio]{../recoloured-tuxscribe-hieroglyphs/png/D35} & ni & nꞽ & D35 \\ 
		\includegraphics[width=0.40\linewidth,height=0.40\linewidth,keepaspectratio]{../recoloured-tuxscribe-hieroglyphs/png/O49} & niwt & nꞽwt & O49 \\ 
		\includegraphics[width=0.50\linewidth,height=0.50\linewidth,keepaspectratio]{../recoloured-tuxscribe-hieroglyphs/png/T34} & nm & nm & T34 \\ 
		\includegraphics[width=0.50\linewidth,height=0.50\linewidth,keepaspectratio]{../recoloured-tuxscribe-hieroglyphs/png/T29} & nmt & nmt & T29 \\ 
		\includegraphics[width=0.50\linewidth,height=0.50\linewidth,keepaspectratio]{../recoloured-tuxscribe-hieroglyphs/png/M22A} & nn & nn & M22a \\ 
		\includegraphics[width=0.50\linewidth,height=0.50\linewidth,keepaspectratio]{../recoloured-tuxscribe-hieroglyphs/png/H4} & nr & nr & H4 \\ 
		\includegraphics[width=0.50\linewidth,height=0.50\linewidth,keepaspectratio]{../recoloured-tuxscribe-hieroglyphs/png/F20} & ns & ns & F20 \\ 
		\includegraphics[width=0.50\linewidth,height=0.50\linewidth,keepaspectratio]{../recoloured-tuxscribe-hieroglyphs/png/K6} & nSmt & nšmt & K6 \\ 
	\end{tabularx}
\end{center}


\begin{center}
	\begin{tabularx}{\linewidth}{YYYY}
		Hieroglyph & MdC+ & Transliteration & Gardiner code\\
		\hline\\
		\includegraphics[width=0.50\linewidth,height=0.50\linewidth,keepaspectratio]{../recoloured-tuxscribe-hieroglyphs/png/R8} & nTr & nṯr & R8 \\ 
		\includegraphics[width=0.50\linewidth,height=0.50\linewidth,keepaspectratio]{../recoloured-tuxscribe-hieroglyphs/png/R8A} & nTrw & nṯrw & R8a \\ 
		\includegraphics[width=0.40\linewidth,height=0.40\linewidth,keepaspectratio]{../recoloured-tuxscribe-hieroglyphs/png/W24} & nw & nw & W24 \\ 
		\includegraphics[width=0.50\linewidth,height=0.50\linewidth,keepaspectratio]{../recoloured-tuxscribe-hieroglyphs/png/O47} & nxn & nḫn & O47 \\ 
		\includegraphics[width=0.50\linewidth,height=0.50\linewidth,keepaspectratio]{../recoloured-tuxscribe-hieroglyphs/png/S45} & nxxw & nḫḫw & S45 \\ 
		\includegraphics[width=0.35\linewidth,height=0.35\linewidth,keepaspectratio]{../recoloured-tuxscribe-hieroglyphs/png/W11} & nzt & nzt & W11 \\ 
		\includegraphics[width=0.50\linewidth,height=0.50\linewidth,keepaspectratio]{../recoloured-tuxscribe-hieroglyphs/png/G47} & pA & pꜣ & G47 \\ 
		\includegraphics[width=0.50\linewidth,height=0.50\linewidth,keepaspectratio]{../recoloured-tuxscribe-hieroglyphs/png/H3} & pAq & pꜣq & H3 \\ 
		\includegraphics[width=0.50\linewidth,height=0.50\linewidth,keepaspectratio]{../recoloured-tuxscribe-hieroglyphs/png/T9} & pd & pd & T9 \\ 
		\includegraphics[width=0.50\linewidth,height=0.50\linewidth,keepaspectratio]{../recoloured-tuxscribe-hieroglyphs/png/T10} & pD & pḏ & T10 \\ 
		\includegraphics[width=0.50\linewidth,height=0.50\linewidth,keepaspectratio]{../recoloured-tuxscribe-hieroglyphs/png/F22} & pH & pḥ & F22 \\ 
		\includegraphics[width=0.50\linewidth,height=0.50\linewidth,keepaspectratio]{../recoloured-tuxscribe-hieroglyphs/png/H2} & pq & pq & H2 \\ 
		\includegraphics[width=0.50\linewidth,height=0.50\linewidth,keepaspectratio]{../recoloured-tuxscribe-hieroglyphs/png/O1} & pr & pr & O1 \\ 
	\end{tabularx}
\end{center}


\begin{center}
	\begin{tabularx}{\linewidth}{YYYY}
		Hieroglyph & MdC+ & Transliteration & Gardiner code\\
		\hline\\
		\includegraphics[width=0.50\linewidth,height=0.50\linewidth,keepaspectratio]{../recoloured-tuxscribe-hieroglyphs/png/N1} & pt & pt & N1 \\ 
		\includegraphics[width=0.50\linewidth,height=0.50\linewidth,keepaspectratio]{../recoloured-tuxscribe-hieroglyphs/png/F46} & pXr & pẖr & F46 \\ 
		\includegraphics[width=0.50\linewidth,height=0.50\linewidth,keepaspectratio]{../recoloured-tuxscribe-hieroglyphs/png/N9} & pzD & pzḏ & N9 \\ 
		\includegraphics[width=0.50\linewidth,height=0.50\linewidth,keepaspectratio]{../recoloured-tuxscribe-hieroglyphs/png/F46} & qAb & qꜣb & F46 \\ 
		\includegraphics[width=0.50\linewidth,height=0.50\linewidth,keepaspectratio]{../recoloured-tuxscribe-hieroglyphs/png/A38} & qiz & qꞽz & A38 \\ 
		\includegraphics[width=0.50\linewidth,height=0.50\linewidth,keepaspectratio]{../recoloured-tuxscribe-hieroglyphs/png/T14} & qmA & qmꜣ & T14 \\ 
		\includegraphics[width=0.50\linewidth,height=0.50\linewidth,keepaspectratio]{../recoloured-tuxscribe-hieroglyphs/png/O38} & qnbt & qnbt & O38 \\ 
		\includegraphics[width=0.50\linewidth,height=0.50\linewidth,keepaspectratio]{../recoloured-tuxscribe-hieroglyphs/png/Q6} & qrsw & qrsw & Q6 \\ 
		\includegraphics[width=0.50\linewidth,height=0.50\linewidth,keepaspectratio]{../recoloured-tuxscribe-hieroglyphs/png/T19} & qs & qs & T19 \\ 
		\includegraphics[width=0.30\linewidth,height=0.30\linewidth,keepaspectratio]{../recoloured-tuxscribe-hieroglyphs/png/N5} & ra & rꜥ & N5 \\ 
		\includegraphics[width=0.30\linewidth,height=0.30\linewidth,keepaspectratio]{../recoloured-tuxscribe-hieroglyphs/png/N5} & rA & rꜣ & N5 \\ 
		\includegraphics[width=0.50\linewidth,height=0.50\linewidth,keepaspectratio]{../recoloured-tuxscribe-hieroglyphs/png/D56} & rd & rd & D56 \\ 
		\includegraphics[width=0.50\linewidth,height=0.50\linewidth,keepaspectratio]{../recoloured-tuxscribe-hieroglyphs/png/X8} & rdi & rdꞽ & X8 \\ 
	\end{tabularx}
\end{center}


\begin{center}
	\begin{tabularx}{\linewidth}{YYYY}
		Hieroglyph & MdC+ & Transliteration & Gardiner code\\
		\hline\\
		\includegraphics[width=0.50\linewidth,height=0.50\linewidth,keepaspectratio]{../recoloured-tuxscribe-hieroglyphs/png/D9} & rmi & rmꞽ & D9 \\ 
		\includegraphics[width=0.50\linewidth,height=0.50\linewidth,keepaspectratio]{../recoloured-tuxscribe-hieroglyphs/png/M4} & rnp & rnp & M4 \\ 
		\includegraphics[width=0.50\linewidth,height=0.50\linewidth,keepaspectratio]{../recoloured-tuxscribe-hieroglyphs/png/T13} & rs & rs & T13 \\ 
		\includegraphics[width=0.50\linewidth,height=0.50\linewidth,keepaspectratio]{../recoloured-tuxscribe-hieroglyphs/png/M24} & rsw & rsw & M24 \\ 
		\includegraphics[width=0.50\linewidth,height=0.50\linewidth,keepaspectratio]{../recoloured-tuxscribe-hieroglyphs/png/U31} & rtH & rtḥ & U31 \\ 
		\vspace{0.30000cm} \includegraphics[width=0.50\linewidth,height=0.50\linewidth,keepaspectratio]{../recoloured-tuxscribe-hieroglyphs/png/E23} \vspace{0.30000cm} & rw & rw & E23 \\ 
		\includegraphics[width=0.50\linewidth,height=0.50\linewidth,keepaspectratio]{../recoloured-tuxscribe-hieroglyphs/png/T12} & rwd & rwd & T12 \\ 
		\includegraphics[width=0.50\linewidth,height=0.50\linewidth,keepaspectratio]{../recoloured-tuxscribe-hieroglyphs/png/T12} & rwD & rwḏ & T12 \\ 
		\includegraphics[width=0.50\linewidth,height=0.50\linewidth,keepaspectratio]{../recoloured-tuxscribe-hieroglyphs/png/G23} & rxyt & rḫyt & G23 \\ 
		\includegraphics[width=0.50\linewidth,height=0.50\linewidth,keepaspectratio]{../recoloured-tuxscribe-hieroglyphs/png/J17} & sA & sꜣ & Aa17 \\ 
		\includegraphics[width=0.50\linewidth,height=0.50\linewidth,keepaspectratio]{../recoloured-tuxscribe-hieroglyphs/png/M8} & SA & šꜣ & M8 \\ 
		\includegraphics[width=0.50\linewidth,height=0.50\linewidth,keepaspectratio]{../recoloured-tuxscribe-hieroglyphs/png/D61} & sAH & sꜣḥ & D61 \\ 
		\includegraphics[width=0.50\linewidth,height=0.50\linewidth,keepaspectratio]{../recoloured-tuxscribe-hieroglyphs/png/I5} & sAq & sꜣq & I5 \\ 
	\end{tabularx}
\end{center}


\begin{center}
	\begin{tabularx}{\linewidth}{YYYY}
		Hieroglyph & MdC+ & Transliteration & Gardiner code\\
		\hline\\
		\includegraphics[width=0.50\linewidth,height=0.50\linewidth,keepaspectratio]{../recoloured-tuxscribe-hieroglyphs/png/N14} & sbA & sbꜣ & N14 \\ 
		\includegraphics[width=0.50\linewidth,height=0.50\linewidth,keepaspectratio]{../recoloured-tuxscribe-hieroglyphs/png/I4} & sbk & sbk & I4 \\ 
		\includegraphics[width=0.50\linewidth,height=0.50\linewidth,keepaspectratio]{../recoloured-tuxscribe-hieroglyphs/png/D56} & sbq & sbq & D56 \\ 
		\includegraphics[width=0.50\linewidth,height=0.50\linewidth,keepaspectratio]{../recoloured-tuxscribe-hieroglyphs/png/F33} & sd & sd & F33 \\ 
		\includegraphics[width=0.50\linewidth,height=0.50\linewidth,keepaspectratio]{../recoloured-tuxscribe-hieroglyphs/png/F30} & Sd & šd & F30 \\ 
		\includegraphics[width=0.50\linewidth,height=0.50\linewidth,keepaspectratio]{../recoloured-tuxscribe-hieroglyphs/png/S19} & sDAw & sḏꜣw & S19 \\ 
		\includegraphics[width=0.30\linewidth,height=0.30\linewidth,keepaspectratio]{../recoloured-tuxscribe-hieroglyphs/png/F21} & sDm & sḏm & F21 \\ 
		\includegraphics[width=0.50\linewidth,height=0.50\linewidth,keepaspectratio]{../recoloured-tuxscribe-hieroglyphs/png/S30} & sf & sf & S30 \\ 
		\includegraphics[width=0.50\linewidth,height=0.50\linewidth,keepaspectratio]{../recoloured-tuxscribe-hieroglyphs/png/S32} & siA & sꞽꜣ & S32 \\ 
		\includegraphics[width=0.50\linewidth,height=0.50\linewidth,keepaspectratio]{../recoloured-tuxscribe-hieroglyphs/png/V29} & sk & sk & V29 \\ 
		\includegraphics[width=0.50\linewidth,height=0.50\linewidth,keepaspectratio]{../recoloured-tuxscribe-hieroglyphs/png/M21} & sm & sm & M21 \\ 
		\includegraphics[width=0.50\linewidth,height=0.50\linewidth,keepaspectratio]{../recoloured-tuxscribe-hieroglyphs/png/N40} & Sm & šm & N40 \\ 
		\includegraphics[width=0.50\linewidth,height=0.50\linewidth,keepaspectratio]{../recoloured-tuxscribe-hieroglyphs/png/M26} & Sma & šmꜥ & M26 \\ 
	\end{tabularx}
\end{center}


\begin{center}
	\begin{tabularx}{\linewidth}{YYYY}
		Hieroglyph & MdC+ & Transliteration & Gardiner code\\
		\hline\\
		\includegraphics[width=0.50\linewidth,height=0.50\linewidth,keepaspectratio]{../recoloured-tuxscribe-hieroglyphs/png/T18} & Sms & šms & T18 \\ 
		\includegraphics[width=0.50\linewidth,height=0.50\linewidth,keepaspectratio]{../recoloured-tuxscribe-hieroglyphs/png/T22} & sn & sn & T22 \\ 
		\includegraphics[width=0.50\linewidth,height=0.50\linewidth,keepaspectratio]{../recoloured-tuxscribe-hieroglyphs/png/V7} & Sn & šn & V7 \\ 
		\includegraphics[width=0.50\linewidth,height=0.50\linewidth,keepaspectratio]{../recoloured-tuxscribe-hieroglyphs/png/U13} & Sna & šnꜥ & U13 \\ 
		\includegraphics[width=0.50\linewidth,height=0.50\linewidth,keepaspectratio]{../recoloured-tuxscribe-hieroglyphs/png/G54} & snD & snḏ & G54 \\ 
		\includegraphics[width=0.50\linewidth,height=0.50\linewidth,keepaspectratio]{../recoloured-tuxscribe-hieroglyphs/png/S26} & Sndyt & šndyt & S26 \\ 
		\includegraphics[width=0.50\linewidth,height=0.50\linewidth,keepaspectratio]{../recoloured-tuxscribe-hieroglyphs/png/V5} & snT & snṯ & V5 \\ 
		\includegraphics[width=0.50\linewidth,height=0.50\linewidth,keepaspectratio]{../recoloured-tuxscribe-hieroglyphs/png/R7} & snTr & snṯr & R7 \\ 
		\includegraphics[width=0.50\linewidth,height=0.50\linewidth,keepaspectratio]{../recoloured-tuxscribe-hieroglyphs/png/O51} & Snwt & šnwt & O51 \\ 
		\includegraphics[width=0.50\linewidth,height=0.50\linewidth,keepaspectratio]{../recoloured-tuxscribe-hieroglyphs/png/D3} & Sny & šny & D3 \\ 
		\includegraphics[width=0.50\linewidth,height=0.50\linewidth,keepaspectratio]{../recoloured-tuxscribe-hieroglyphs/png/N24} & spAt & spꜣt & N24 \\ 
		\includegraphics[width=0.50\linewidth,height=0.50\linewidth,keepaspectratio]{../recoloured-tuxscribe-hieroglyphs/png/F42} & spr & spr & F42 \\ 
		\includegraphics[width=0.50\linewidth,height=0.50\linewidth,keepaspectratio]{../recoloured-tuxscribe-hieroglyphs/png/A50} & Sps & šps & A50 \\ 
	\end{tabularx}
\end{center}


\begin{center}
	\begin{tabularx}{\linewidth}{YYYY}
		Hieroglyph & MdC+ & Transliteration & Gardiner code\\
		\hline\\
		\includegraphics[width=0.50\linewidth,height=0.50\linewidth,keepaspectratio]{../recoloured-tuxscribe-hieroglyphs/png/A51} & Spsi & špsꞽ & A51 \\ 
		\includegraphics[width=0.50\linewidth,height=0.50\linewidth,keepaspectratio]{../recoloured-tuxscribe-hieroglyphs/png/D24} & spt & spt & D24 \\ 
		\includegraphics[width=0.50\linewidth,height=0.50\linewidth,keepaspectratio]{../recoloured-tuxscribe-hieroglyphs/png/D25} & spty & spty & D25 \\ 
		\includegraphics[width=0.50\linewidth,height=0.50\linewidth,keepaspectratio]{../recoloured-tuxscribe-hieroglyphs/png/A21} & sr & sr & A21 \\ 
		\includegraphics[width=0.50\linewidth,height=0.50\linewidth,keepaspectratio]{../recoloured-tuxscribe-hieroglyphs/png/L7} & srqt & srqt & L7 \\ 
		\includegraphics[width=0.50\linewidth,height=0.50\linewidth,keepaspectratio]{../recoloured-tuxscribe-hieroglyphs/png/V6} & sS & sš & V6 \\ 
		\includegraphics[width=0.50\linewidth,height=0.50\linewidth,keepaspectratio]{../recoloured-tuxscribe-hieroglyphs/png/F5} & SsA & šsꜣ & F5 \\ 
		\includegraphics[width=0.50\linewidth,height=0.50\linewidth,keepaspectratio]{../recoloured-tuxscribe-hieroglyphs/png/T31} & sSm & sšm & T31 \\ 
		\includegraphics[width=0.35\linewidth,height=0.35\linewidth,keepaspectratio]{../recoloured-tuxscribe-hieroglyphs/png/V33} & sSr & sšr & V33 \\ 
		\includegraphics[width=0.50\linewidth,height=0.50\linewidth,keepaspectratio]{../recoloured-tuxscribe-hieroglyphs/png/Q1} & st & st & Q1 \\ 
		\includegraphics[width=0.50\linewidth,height=0.50\linewidth,keepaspectratio]{../recoloured-tuxscribe-hieroglyphs/png/S22} & sT & sṯ & S22 \\ 
		\includegraphics[width=0.50\linewidth,height=0.50\linewidth,keepaspectratio]{../recoloured-tuxscribe-hieroglyphs/png/V2} & sTA & sṯꜣ & V2 \\ 
		\includegraphics[width=0.50\linewidth,height=0.50\linewidth,keepaspectratio]{../recoloured-tuxscribe-hieroglyphs/png/V3} & sTAw & sṯꜣw & V3 \\ 
	\end{tabularx}
\end{center}


\begin{center}
	\begin{tabularx}{\linewidth}{YYYY}
		Hieroglyph & MdC+ & Transliteration & Gardiner code\\
		\hline\\
		\includegraphics[width=0.50\linewidth,height=0.50\linewidth,keepaspectratio]{../recoloured-tuxscribe-hieroglyphs/png/F29} & sti & stꞽ & F29 \\ 
		\includegraphics[width=0.50\linewidth,height=0.50\linewidth,keepaspectratio]{../recoloured-tuxscribe-hieroglyphs/png/U21} & stp & stp & U21 \\ 
		\includegraphics[width=0.50\linewidth,height=0.50\linewidth,keepaspectratio]{../recoloured-tuxscribe-hieroglyphs/png/C7} & stX & stẖ & C7 \\ 
		\includegraphics[width=0.50\linewidth,height=0.50\linewidth,keepaspectratio]{../recoloured-tuxscribe-hieroglyphs/png/I2} & Styw & štyw & I2 \\ 
		\includegraphics[width=0.50\linewidth,height=0.50\linewidth,keepaspectratio]{../recoloured-tuxscribe-hieroglyphs/png/M23} & sw & sw & M23 \\ 
		\includegraphics[width=0.50\linewidth,height=0.50\linewidth,keepaspectratio]{../recoloured-tuxscribe-hieroglyphs/png/H6} & Sw & šw & H6 \\ 
		\includegraphics[width=0.50\linewidth,height=0.50\linewidth,keepaspectratio]{../recoloured-tuxscribe-hieroglyphs/png/S35} & Swt & šwt & S35 \\ 
		\includegraphics[width=0.50\linewidth,height=0.50\linewidth,keepaspectratio]{../recoloured-tuxscribe-hieroglyphs/png/S9} & Swty & šwty & S9 \\ 
		\includegraphics[width=0.50\linewidth,height=0.50\linewidth,keepaspectratio]{../recoloured-tuxscribe-hieroglyphs/png/S42} & sxm & sḫm & S42 \\ 
		\includegraphics[width=0.50\linewidth,height=0.50\linewidth,keepaspectratio]{../recoloured-tuxscribe-hieroglyphs/png/S6} & sxmty & sḫmty & S6 \\ 
		\includegraphics[width=0.50\linewidth,height=0.50\linewidth,keepaspectratio]{../recoloured-tuxscribe-hieroglyphs/png/T11} & sXr & sẖr & T11 \\ 
		\includegraphics[width=0.50\linewidth,height=0.50\linewidth,keepaspectratio]{../recoloured-tuxscribe-hieroglyphs/png/M20} & sxt & sḫt & M20 \\ 
		\includegraphics[width=0.50\linewidth,height=0.50\linewidth,keepaspectratio]{../recoloured-tuxscribe-hieroglyphs/png/O42} & Szp & šzp & O42 \\ 
	\end{tabularx}
\end{center}


\begin{center}
	\begin{tabularx}{\linewidth}{YYYY}
		Hieroglyph & MdC+ & Transliteration & Gardiner code\\
		\hline\\
		\includegraphics[width=0.50\linewidth,height=0.50\linewidth,keepaspectratio]{../recoloured-tuxscribe-hieroglyphs/png/N16} & tA & tꜣ & N16 \\ 
		\includegraphics[width=0.50\linewidth,height=0.50\linewidth,keepaspectratio]{../recoloured-tuxscribe-hieroglyphs/png/G47} & TA & ṯꜣ & G47 \\ 
		\includegraphics[width=0.50\linewidth,height=0.50\linewidth,keepaspectratio]{../recoloured-tuxscribe-hieroglyphs/png/P5} & TAw & ṯꜣw & P5 \\ 
		\includegraphics[width=0.50\linewidth,height=0.50\linewidth,keepaspectratio]{../recoloured-tuxscribe-hieroglyphs/png/S33} & Tb & ṯb & S33 \\ 
		\includegraphics[width=0.50\linewidth,height=0.50\linewidth,keepaspectratio]{../recoloured-tuxscribe-hieroglyphs/png/S15} & tHn & tḥn & S15 \\ 
		\includegraphics[width=0.50\linewidth,height=0.50\linewidth,keepaspectratio]{../recoloured-tuxscribe-hieroglyphs/png/S15} & THn & ṯḥn & S15 \\ 
		\includegraphics[width=0.50\linewidth,height=0.50\linewidth,keepaspectratio]{../recoloured-tuxscribe-hieroglyphs/png/U33} & ti & tꞽ & U33 \\ 
		\includegraphics[width=0.50\linewidth,height=0.50\linewidth,keepaspectratio]{../recoloured-tuxscribe-hieroglyphs/png/U15} & tm & tm & U15 \\ 
		\includegraphics[width=0.50\linewidth,height=0.50\linewidth,keepaspectratio]{../recoloured-tuxscribe-hieroglyphs/png/V19} & TmA & ṯmꜣ & V19 \\ 
		\includegraphics[width=0.50\linewidth,height=0.50\linewidth,keepaspectratio]{../recoloured-tuxscribe-hieroglyphs/png/D1} & tp & tp & D1 \\ 
		\includegraphics[width=0.50\linewidth,height=0.50\linewidth,keepaspectratio]{../recoloured-tuxscribe-hieroglyphs/png/M6} & tr & tr & M6 \\ 
		\includegraphics[width=0.50\linewidth,height=0.50\linewidth,keepaspectratio]{../recoloured-tuxscribe-hieroglyphs/png/O25} & txn & tḫn & O25 \\ 
		\includegraphics[width=0.50\linewidth,height=0.50\linewidth,keepaspectratio]{../recoloured-tuxscribe-hieroglyphs/png/G4} & tyw & tyw & G4 \\ 
	\end{tabularx}
\end{center}


\begin{center}
	\begin{tabularx}{\linewidth}{YYYY}
		Hieroglyph & MdC+ & Transliteration & Gardiner code\\
		\hline\\
		\includegraphics[width=0.50\linewidth,height=0.50\linewidth,keepaspectratio]{../recoloured-tuxscribe-hieroglyphs/png/S24} & Tz & ṯz & S24 \\ 
		\includegraphics[width=0.50\linewidth,height=0.50\linewidth,keepaspectratio]{../recoloured-tuxscribe-hieroglyphs/png/T21} & wa & wꜥ & T21 \\ 
		\includegraphics[width=0.50\linewidth,height=0.50\linewidth,keepaspectratio]{../recoloured-tuxscribe-hieroglyphs/png/V4} & wA & wꜣ & V4 \\ 
		\includegraphics[width=0.50\linewidth,height=0.50\linewidth,keepaspectratio]{../recoloured-tuxscribe-hieroglyphs/png/D60} & wab & wꜥb & D60 \\ 
		\includegraphics[width=0.50\linewidth,height=0.50\linewidth,keepaspectratio]{../recoloured-tuxscribe-hieroglyphs/png/M13} & wAD & wꜣḏ & M13 \\ 
		\includegraphics[width=0.50\linewidth,height=0.50\linewidth,keepaspectratio]{../recoloured-tuxscribe-hieroglyphs/png/V29} & wAH & wꜣḥ & V29 \\ 
		\includegraphics[width=0.50\linewidth,height=0.50\linewidth,keepaspectratio]{../recoloured-tuxscribe-hieroglyphs/png/S40} & wAs & wꜣs & S40 \\ 
		\includegraphics[width=0.50\linewidth,height=0.50\linewidth,keepaspectratio]{../recoloured-tuxscribe-hieroglyphs/png/U26} & wbA & wbꜣ & U26 \\ 
		\includegraphics[width=0.50\linewidth,height=0.50\linewidth,keepaspectratio]{../recoloured-tuxscribe-hieroglyphs/png/V24} & wD & wḏ & V24 \\ 
		\includegraphics[width=0.50\linewidth,height=0.50\linewidth,keepaspectratio]{../recoloured-tuxscribe-hieroglyphs/png/M13} & WD & wḏ & M13 \\ 
		\includegraphics[width=0.50\linewidth,height=0.50\linewidth,keepaspectratio]{../recoloured-tuxscribe-hieroglyphs/png/U28} & wDA & wḏꜣ & U28 \\ 
		\includegraphics[width=0.50\linewidth,height=0.50\linewidth,keepaspectratio]{../recoloured-tuxscribe-hieroglyphs/png/D10} & wDAt & wḏꜣt & D10 \\ 
		\includegraphics[width=0.50\linewidth,height=0.50\linewidth,keepaspectratio]{../recoloured-tuxscribe-hieroglyphs/png/N20} & wDb & wḏb & N20 \\ 
	\end{tabularx}
\end{center}


\begin{center}
	\begin{tabularx}{\linewidth}{YYYY}
		Hieroglyph & MdC+ & Transliteration & Gardiner code\\
		\hline\\
		\includegraphics[width=0.50\linewidth,height=0.50\linewidth,keepaspectratio]{../recoloured-tuxscribe-hieroglyphs/png/M11} & wdn & wdn & M11 \\ 
		\includegraphics[width=0.50\linewidth,height=0.50\linewidth,keepaspectratio]{../recoloured-tuxscribe-hieroglyphs/png/P4} & wHa & wḥꜥ & P4 \\ 
		\includegraphics[width=0.50\linewidth,height=0.50\linewidth,keepaspectratio]{../recoloured-tuxscribe-hieroglyphs/png/F25} & wHm & wḥm & F25 \\ 
		\includegraphics[width=0.50\linewidth,height=0.50\linewidth,keepaspectratio]{../recoloured-tuxscribe-hieroglyphs/png/E34} & wn & wn & E34 \\ 
		\includegraphics[width=0.50\linewidth,height=0.50\linewidth,keepaspectratio]{../recoloured-tuxscribe-hieroglyphs/png/F13} & wp & wp & F13 \\ 
		\includegraphics[width=0.50\linewidth,height=0.50\linewidth,keepaspectratio]{../recoloured-tuxscribe-hieroglyphs/png/G36} & wr & wr & G36 \\ 
		\includegraphics[width=0.50\linewidth,height=0.50\linewidth,keepaspectratio]{../recoloured-tuxscribe-hieroglyphs/png/T17} & wrrt & wrrt & T17 \\ 
		\includegraphics[width=0.50\linewidth,height=0.50\linewidth,keepaspectratio]{../recoloured-tuxscribe-hieroglyphs/png/G42} & wSA & wšꜣ & G42 \\ 
		\includegraphics[width=0.50\linewidth,height=0.50\linewidth,keepaspectratio]{../recoloured-tuxscribe-hieroglyphs/png/H2} & wSm & wšm & H2 \\ 
		\includegraphics[width=0.50\linewidth,height=0.50\linewidth,keepaspectratio]{../recoloured-tuxscribe-hieroglyphs/png/F12} & wsr & wsr & F12 \\ 
		\includegraphics[width=0.50\linewidth,height=0.50\linewidth,keepaspectratio]{../recoloured-tuxscribe-hieroglyphs/png/S11} & wsx & wsḫ & S11 \\ 
		\includegraphics[width=0.50\linewidth,height=0.50\linewidth,keepaspectratio]{../recoloured-tuxscribe-hieroglyphs/png/O15} & wsxt & wsḫt & O15 \\ 
		\includegraphics[width=0.50\linewidth,height=0.50\linewidth,keepaspectratio]{../recoloured-tuxscribe-hieroglyphs/png/G44} & ww & ww & G44 \\ 
	\end{tabularx}
\end{center}


\begin{center}
	\begin{tabularx}{\linewidth}{YYYY}
		Hieroglyph & MdC+ & Transliteration & Gardiner code\\
		\hline\\
		\includegraphics[width=0.50\linewidth,height=0.50\linewidth,keepaspectratio]{../recoloured-tuxscribe-hieroglyphs/png/R16} & wx & wḫ & R16 \\ 
		\includegraphics[width=0.50\linewidth,height=0.50\linewidth,keepaspectratio]{../recoloured-tuxscribe-hieroglyphs/png/Q2} & wz & wz & Q2 \\ 
		\includegraphics[width=0.50\linewidth,height=0.50\linewidth,keepaspectratio]{../recoloured-tuxscribe-hieroglyphs/png/N28} & xa & ḫꜥ & N28 \\ 
		\includegraphics[width=0.50\linewidth,height=0.50\linewidth,keepaspectratio]{../recoloured-tuxscribe-hieroglyphs/png/M12} & xA & ḫꜣ & M12 \\ 
		\includegraphics[width=0.50\linewidth,height=0.50\linewidth,keepaspectratio]{../recoloured-tuxscribe-hieroglyphs/png/K4} & XA & ẖꜣ & K4 \\ 
		\includegraphics[width=0.50\linewidth,height=0.50\linewidth,keepaspectratio]{../recoloured-tuxscribe-hieroglyphs/png/V19} & XAr & ẖꜣr & V19 \\ 
		\includegraphics[width=0.50\linewidth,height=0.50\linewidth,keepaspectratio]{../recoloured-tuxscribe-hieroglyphs/png/N25} & xAst & ḫꜣst & N25 \\ 
		\includegraphics[width=0.50\linewidth,height=0.50\linewidth,keepaspectratio]{../recoloured-tuxscribe-hieroglyphs/png/R1} & xAt & ḫꜣt & R1 \\ 
		\includegraphics[width=0.50\linewidth,height=0.50\linewidth,keepaspectratio]{../recoloured-tuxscribe-hieroglyphs/png/R1} & xAwt & ḫꜣwt & R1 \\ 
		\includegraphics[width=0.50\linewidth,height=0.50\linewidth,keepaspectratio]{../recoloured-tuxscribe-hieroglyphs/png/S1} & xDt & ḫḏt & S1 \\ 
		\includegraphics[width=0.50\linewidth,height=0.50\linewidth,keepaspectratio]{../recoloured-tuxscribe-hieroglyphs/png/U36} & xm & ḫm & U36 \\ 
		\includegraphics[width=0.50\linewidth,height=0.50\linewidth,keepaspectratio]{../recoloured-tuxscribe-hieroglyphs/png/G41} & xn & ḫn & G41 \\ 
		\includegraphics[width=0.50\linewidth,height=0.50\linewidth,keepaspectratio]{../recoloured-tuxscribe-hieroglyphs/png/F26} & Xn & ẖn & F26 \\ 
	\end{tabularx}
\end{center}


\begin{center}
	\begin{tabularx}{\linewidth}{YYYY}
		Hieroglyph & MdC+ & Transliteration & Gardiner code\\
		\hline\\
		\includegraphics[width=0.50\linewidth,height=0.50\linewidth,keepaspectratio]{../recoloured-tuxscribe-hieroglyphs/png/W9} & Xnm & ẖnm & W9 \\ 
		\includegraphics[width=0.50\linewidth,height=0.50\linewidth,keepaspectratio]{../recoloured-tuxscribe-hieroglyphs/png/C4} & Xnmw & ẖnmw & C4 \\ 
		\includegraphics[width=0.50\linewidth,height=0.50\linewidth,keepaspectratio]{../recoloured-tuxscribe-hieroglyphs/png/W17} & xnt & ḫnt & W17 \\ 
		\includegraphics[width=0.50\linewidth,height=0.50\linewidth,keepaspectratio]{../recoloured-tuxscribe-hieroglyphs/png/L1} & xpr & ḫpr & L1 \\ 
		\includegraphics[width=0.50\linewidth,height=0.50\linewidth,keepaspectratio]{../recoloured-tuxscribe-hieroglyphs/png/S7} & xprS & ḫprš & S7 \\ 
		\includegraphics[width=0.50\linewidth,height=0.50\linewidth,keepaspectratio]{../recoloured-tuxscribe-hieroglyphs/png/F23} & xpS & ḫpš & F23 \\ 
		\includegraphics[width=0.50\linewidth,height=0.50\linewidth,keepaspectratio]{../recoloured-tuxscribe-hieroglyphs/png/A15} & xr & ḫr & A15 \\ 
		\includegraphics[width=0.50\linewidth,height=0.50\linewidth,keepaspectratio]{../recoloured-tuxscribe-hieroglyphs/png/T28} & Xr & ẖr & T28 \\ 
		\includegraphics[width=0.50\linewidth,height=0.50\linewidth,keepaspectratio]{../recoloured-tuxscribe-hieroglyphs/png/A17} & Xrd & ẖrd & A17 \\ 
		\includegraphics[width=0.50\linewidth,height=0.50\linewidth,keepaspectratio]{../recoloured-tuxscribe-hieroglyphs/png/S42} & xrp & ḫrp & S42 \\ 
		\includegraphics[width=0.50\linewidth,height=0.50\linewidth,keepaspectratio]{../recoloured-tuxscribe-hieroglyphs/png/P8} & xrw & ḫrw & P8 \\ 
		\includegraphics[width=0.50\linewidth,height=0.50\linewidth,keepaspectratio]{../recoloured-tuxscribe-hieroglyphs/png/U34} & xsf & ḫsf & U34 \\ 
		\includegraphics[width=0.50\linewidth,height=0.50\linewidth,keepaspectratio]{../recoloured-tuxscribe-hieroglyphs/png/M3} & xt & ḫt & M3 \\ 
	\end{tabularx}
\end{center}


\begin{center}
	\begin{tabularx}{\linewidth}{YYYY}
		Hieroglyph & MdC+ & Transliteration & Gardiner code\\
		\hline\\
		\includegraphics[width=0.50\linewidth,height=0.50\linewidth,keepaspectratio]{../recoloured-tuxscribe-hieroglyphs/png/S20} & xtm & ḫtm & S20 \\ 
		\includegraphics[width=0.50\linewidth,height=0.50\linewidth,keepaspectratio]{../recoloured-tuxscribe-hieroglyphs/png/S37} & xw & ḫw & S37 \\ 
		\includegraphics[width=0.50\linewidth,height=0.50\linewidth,keepaspectratio]{../recoloured-tuxscribe-hieroglyphs/png/G39} & zA & zꜣ & G39 \\ 
		\includegraphics[width=0.50\linewidth,height=0.50\linewidth,keepaspectratio]{../recoloured-tuxscribe-hieroglyphs/png/E17} & zAb & zꜣb & E17 \\ 
		\includegraphics[width=0.50\linewidth,height=0.50\linewidth,keepaspectratio]{../recoloured-tuxscribe-hieroglyphs/png/O35} & zb & zb & O35 \\ 
		\includegraphics[width=0.50\linewidth,height=0.50\linewidth,keepaspectratio]{../recoloured-tuxscribe-hieroglyphs/png/O22} & zH & zḥ & O22 \\ 
		\includegraphics[width=0.50\linewidth,height=0.50\linewidth,keepaspectratio]{../recoloured-tuxscribe-hieroglyphs/png/T11} & zin & zꞽn & T11 \\ 
		\includegraphics[width=0.50\linewidth,height=0.50\linewidth,keepaspectratio]{../recoloured-tuxscribe-hieroglyphs/png/F36} & zmA & zmꜣ & F36 \\ 
		\includegraphics[width=0.50\linewidth,height=0.50\linewidth,keepaspectratio]{../recoloured-tuxscribe-hieroglyphs/png/U32} & zmn & zmn & U32 \\ 
		\includegraphics[width=0.50\linewidth,height=0.50\linewidth,keepaspectratio]{../recoloured-tuxscribe-hieroglyphs/png/O50} & zp & zp & O50 \\ 
		\includegraphics[width=0.50\linewidth,height=0.50\linewidth,keepaspectratio]{../recoloured-tuxscribe-hieroglyphs/png/Y3} & zS & zš & Y3 \\ 
		\includegraphics[width=0.50\linewidth,height=0.50\linewidth,keepaspectratio]{../recoloured-tuxscribe-hieroglyphs/png/M9} & zSn & zšn & M9 \\ 
		\includegraphics[width=0.50\linewidth,height=0.50\linewidth,keepaspectratio]{../recoloured-tuxscribe-hieroglyphs/png/Y8} & zSSt & zššt & Y8 \\ 
	\end{tabularx}
\end{center}


\begin{center}
	\begin{tabularx}{\linewidth}{YYYY}
		Hieroglyph & MdC+ & Transliteration & Gardiner code\\
		\hline\\
		\includegraphics[width=0.30\linewidth,height=0.30\linewidth,keepaspectratio]{../recoloured-tuxscribe-hieroglyphs/png/N5} & zw & zw & N5 \\ 
		\includegraphics[width=0.50\linewidth,height=0.50\linewidth,keepaspectratio]{../recoloured-tuxscribe-hieroglyphs/png/T11} & zwn & zwn & T11 \\ 
		\includegraphics[width=0.50\linewidth,height=0.50\linewidth,keepaspectratio]{../recoloured-tuxscribe-hieroglyphs/png/Y3} & zx & zḫ & Y3 \\ 
		\includegraphics[width=0.50\linewidth,height=0.50\linewidth,keepaspectratio]{../recoloured-tuxscribe-hieroglyphs/png/O30} & zxnt & zḫnt & O30 \\ 
		\includegraphics[width=0.50\linewidth,height=0.50\linewidth,keepaspectratio]{../recoloured-tuxscribe-hieroglyphs/png/E6} & zzmt & zzmt & E6 \\ 
	\end{tabularx}
\end{center}


\section*{Extended Gardiner sign list}
\markboth{Notation}{Extended Gardiner sign list}
\addcontentsline{toc}{section}{Extended Gardiner sign list}
