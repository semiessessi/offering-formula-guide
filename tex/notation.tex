The transliteration and transcription notation used in this book is a variant of the \textit{\indexed{Manuel de Codage}} (MdC), referring to an extended version of \textit{\indexed{Gardiner's sign list}}.

MdC is an Egyptological standard for computer encoding of hieroglyphic text based on the phonectic values. It was first published in 1988 and continues to be revised.

It uses only a subset of the latin alphabet, with capitalisation and Gardiner sign codes for greater specificity. It also includes markup characters to encode positioning and arrangement of hieroglyphs.

In this book a variation is used which prefers phonetic values to sign codes.

The Gardiner sign list categorises hieroglyphs into types, and assigns each one a unique code. Although a given \indexed{hieroglyph} my fit into multiple categories, they occur only once, and usually in the first applicable category. It was first published in 1927, although the discovery of new glyphs and variants has required its extension since then.

For example, the rendition of hieroglyph G10 portrays a \indexed{falcon}, representing the god \nname{Sokar}, standing on a \indexed{barque} on top of a sled.

(TODO: image...)

It could reasonably fit into the categories G (Birds), P (Ships and parts of ships), R (Temple furniture or sacred emblems) or U (Agriculture, crafts, and professions). It appears only once in the G category.

...

\section*{Manuel de Codage variant}
\markboth{Notation}{Manuel de Codage variant}
\addcontentsline{toc}{section}{Manuel de Codage variant}

\section*{Extended Gardiner sign list}
\markboth{Notation}{Extended Gardiner sign list}
\addcontentsline{toc}{section}{Extended Gardiner sign list}
