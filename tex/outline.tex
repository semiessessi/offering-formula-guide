The shape of the offering formula is always the same. It opens with the same standard phrase, invokes a god to pass along the offering, lists the offerings then names the recipient.

\section*{\indexed{Htp-di-nsw}}
\addcontentsline{toc}{section}{Htp-di-nsw}

\begin{figure} [H]
	\centering
	\includegraphics[width=0.4\textwidth]{../images/htp-di-nsw}
	\caption{Htp-di-nsw as commonly rendered in hieroglyphs}
\end{figure}

Many renditions of the offering formula begin with this compound expression. The exact interpretation is debated, but this is often rendered as "a royal offering" or "an offering given by the king".

This expression is sometimes used to describe an offering formula.

By convention the t from \indexed{nswt} is dropped, although it often appears in writing and inscriptions.

The order of the words when written is a case of \textit{\indexed{honorific transposition}}, with the \indexed{nswt} part written first. This is to show the importance of the king, and also applies to names of gods and the word nTr which loosely translates to "god".

\subsection*{Htp}
\addcontentsline{toc}{subsection}{Htp}

\begin{figure} [H]
	\centering
	\includegraphics[width=0.275\textwidth]{../images/htp}
	\caption{A more complete spelling of \indexed{Htp} - \textbf{H t p} [Htp]}
\end{figure}

The word \indexed{Htp} has no precise translation into English, and is variously rendered as "contentment", "peace" or "offering". This is sufficient to grasp its true meaning, since offerings are intended to bring comfort and bliss.

\begin{figure} [H]
	\centering
	\includegraphics[width=0.175\textwidth]{../images/htp2}
	\caption{Another spelling of \indexed{Htp} - \textbf{Htp} t p [X4]}
\end{figure}

\subsection*{di}
\addcontentsline{toc}{subsection}{di}

\begin{figure} [H]
	\centering
	\includegraphics[width=0.125\textwidth]{../recoloured-tuxscribe-hieroglyphs/png/X8}
	\caption{the (r)di hieroglyph - X8}
\end{figure}

This word is a verb, (r)\indexed{di} - to give, which in older writings is sometimes rendered as \indexed{rdi} rather than di. It also appears later in the formula, but with a suffix pronoun, either .f, .s or .sn depending on the god(s) invoked.

\begin{figure} [H]
	\centering
	\includegraphics[width=0.125\textwidth]{../recoloured-tuxscribe-hieroglyphs/png/D37}
	\caption{the (r)di hieroglyph commonly used when writing di.f - D37}
\end{figure}

\subsection*{nswt}
\addcontentsline{toc}{subsection}{nswt}

The term nswt means the king or ruler. It was treated with great reverence as the embodiment of the institution of statehood, which was unique in its earliest form.

In older Egyptological works it is transliterated as swtn, but it is now believed that the order of the hieroglyphs is a form of \indexed{honorific transposition}.

\section*{prt-xrw}
\addcontentsline{toc}{section}{prt-xrw}

\begin{figure} [H]
	\centering
	\includegraphics[width=0.25\textwidth]{../recoloured-tuxscribe-hieroglyphs/png/O3}
	\caption{the prt-xrw hieroglyph - O3}
\end{figure}

The expression \indexed{prt-xrw} is usually translated as "\indexed{voice offering}", although it more directly means "emerging from voice".

The hieroglyph contains the bread and beer signs, but this is by convention, and doesn't necessarily mean that the voice offering includes bread and beer.

\section*{Offering(s)}
\addcontentsline{toc}{section}{Offerings}

Thee is something of a standard list of offerings, which is usually terminated with the expression xt nbt nfrt wabt Anxt nTr im - all the beautiful and pure things on which a god lives.

\subsection*{Bread - t}
\addcontentsline{toc}{subsection}{Bread - t}

\begin{figure} [H]
	\centering
	\includegraphics[width=0.25\textwidth]{../recoloured-tuxscribe-hieroglyphs/png/X1}
	\caption{the t hieroglyph - X1}
\end{figure}

Bread\index{bread} was a staple in the diet of Ancient Egypt, and bread making and consumption was deeply integrated into their culture.

It was primarily made from emmer \indexed{wheat} or \indexed{barley}, which was ground into \indexed{flour} using stones. The dough, made by mixing with water, and sometimes honey or oil, was left to naturally ferment, giving it a slightly sour flavour. Baking was usually done in conical clay ovens, where the dough was either placed on the oven walls, or baked in molds.

The hieroglyph for (r)di, X8, represents a sacrificial loaf, although it is a highly stylised rendition. The hieroglyph for t, X1, is also a stylised loaf. In the prt-xrw hieroglyph we can see X3 also used to represent a loaf.

\begin{figure} [H]
	\centering
	\includegraphics[width=0.25\textwidth]{../recoloured-tuxscribe-hieroglyphs/png/X3}
	\caption{the bread hieroglyph - X3}
\end{figure}

\subsection*{Beer - Hnqt}
\addcontentsline{toc}{subsection}{Beer - Hnqt}

Beer\index{beer} was an integral part of daily life in ancient Egypt, enjoyed by both the rich and the poor.

It was primarily made from the aforementioned \indexed{bread}, which was crumbled into water and left to ferment. The resulting brew was thick and nutritious, often flavoured with herbs, honey, or fruits.

\subsection*{Oxen - kAw}
\addcontentsline{toc}{subsection}{Oxen - kAw}

\begin{figure} [H]
	\centering
	\includegraphics[width=0.25\textwidth]{../recoloured-tuxscribe-hieroglyphs/png/F1}
	\caption{the kA(w) hieroglyph - F1}
\end{figure}

In ancient Egypt, \indexed{oxen}, castrated \indexed{bull}s, were highly valued for their critical role in agriculture and daily life. These sturdy animals were used for plowing fields, threshing grain, and transporting heavy loads, making them indispensable for farming communities along the Nile.

Bulls\index{bull} and \indexed{cow}s themselves were often part of religious ceremonies and offerings, representing strength and fertility. The \indexed{cow} was sacred to \nname{Hathor}, and sacred bulls were considered incarnations of the gods, including the \nname{Apis}, \nname{Mnevis} and \nname{Buchis} bulls.

Beef seems to have been considered somewhat of a luxury and was typically reserved for the elite and for special occasions.

Tomb paintings and carvings frequently depicted scenes of cattle being tended to, highlighting their importance in both practical and ceremonial contexts.

\subsection*{Fowl - Apdw}
\addcontentsline{toc}{subsection}{Fowl - Apdw}

\begin{figure} [H]
	\centering
	\includegraphics[width=0.25\textwidth]{../recoloured-tuxscribe-hieroglyphs/png/H1}
	\caption{the Apd(w) hieroglyph - H1}
\end{figure}

In ancient Egypt, \indexed{fowl} such as \indexed{duck}s, geese\index{goose}, pigeons, and quails played a significant role in both daily life and religious practices. In particular pigeons were trained as carriers for communication.

These birds were commonly raised in households and on farms, as well as being harvested from the river itself. They provided a source of meat, eggs, and feathers.

The Egyptians employed various techniques to catch wild birds, including netting and trapping. Fowl were often depicted in tomb paintings and reliefs, showcasing their importance in the Egyptian diet and culture.

\subsection*{Alabaster - Ss}
\addcontentsline{toc}{subsection}{Alabaster - Ss}

\begin{figure} [H]
	\centering
	\includegraphics[width=0.25\textwidth]{../recoloured-tuxscribe-hieroglyphs/png/V6}
	\caption{the Ss hieroglyph - V6}
\end{figure}

Alabaster.

\subsection*{Linen - mnxt}
\addcontentsline{toc}{subsection}{Linen - mnxt}

Linen.

\section*{God(s)}
\addcontentsline{toc}{section}{God(s)}

\subsection*{Anubis, who sits on his mountain}
\addcontentsline{toc}{subsection}{Anubis, who sits on his mountain}

\subsection*{Osiris, lord of Abydos}
\addcontentsline{toc}{subsection}{Osiris, lord of Abydos}

\subsection*{Hathor, lady of the west}
\addcontentsline{toc}{subsection}{Hathor, lady of the west}

\section*{n kA n}

The offering formula is directed at the \indexed{ka} of the recipient.

\section*{mAa-xrw}

True of voice.

\section*{Putting it all together}